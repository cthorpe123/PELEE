\section{Introduction}
\label{sec:Introduction}

The PELEE Statistical Update is an upgraded version of the PELEE analysis where we add runs 4 and 5. In addition to the addition of more statistics, some improvements to the event selection, signal modelling and treatment of systematics have also been made that are documented here. 

\subsection{Goals of this analysis}
\label{sec:Goals}
(Giuseppe, work in progress.)

Goal of this analysis is to perform a statistical update of the first pionless eLEE analysis, thus checking with a larger dataset the stability of the results and comparing the data to a signal model that relies less on assumptions about the physics process leading to the excess in MiniBooNE.

Therefore, updates to the analysis are primarily of statistical nature, either in terms of the amount of data included, or in terms of the statistical procedures used to extract the result:
\begin{itemize}
    \item Include the full MicroBooNE BNB data, from run1 to run5.
    \item Update the signal model of the MiniBooNE excess, now based on the electron kinematics.
    \item Update the statistical treatment of the cosmic background (estimated with triggered beam-off, "EXT", events)
    \item Explore potential updates to other statistical procedures, such as correlations for detector variations, and the constraint selection.
\end{itemize}
The only selection update we consider in this update is the usage of the CRT veto in the nue selection for run3-5. This selection was already used in the numu selection (for constraint) in the first round of the analysis and it makes sense to use it also for the nue selection given that all the additional data include the CRT detector.

Further updates to the reconstruction and to the selection will be included in the final version of the analysis, which is planned for 2025.