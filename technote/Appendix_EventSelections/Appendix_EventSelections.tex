\section{Event Selections}

\subsection{Preselections}
\label{appendix:Preselections}

\renewcommand{\arraystretch}{1.4}
\begin{table}[H]
    \centering
    \begin{tabular}{c|c}
        \hline
        \hline
        Cut Name & Query \\
        \hline
        \hline
        Neutrino slice ID & \verb|nslice == 1|  \\
        \hline
        Topological score & \verb|topological_score > 0.06| \\
        \hline
        \multirow{6}{*}{Fiducial volume} & \verb|reco_nu_vtx_sce_x > 5| \\
        & \verb|reco_nu_vtx_sce_x < 251| \\
        & \verb|reco_nu_vtx_sce_y > -110| \\
        & \verb|reco_nu_vtx_sce_y < 110| \\
        & \verb|reco_nu_vtx_sce_z > 20| \\
        & \verb|reco_nu_vtx_sce_z < 986| \\
        \hline
        \end{tabular}
    \caption{Muon neutrino preselection.}
    \label{tab:NuMuPresel}    
\end{table}
\renewcommand{\arraystretch}{1.0}

\renewcommand{\arraystretch}{1.4}
\begin{table}[H]
    \centering
    \begin{tabular}{c|c}
        \hline
        \hline
        Cut Name & Query \\
        \hline
        \hline
        Neutrino slice ID & \verb|nslice == 1|  \\
        \hline
        Selection? & \verb|selected == 1| \\
        \hline
        Shower energy & \verb|shr_energy_tot_cali > 0.07| \\
        \hline
        \end{tabular}
    \caption{Electron neutrino preselection.}
    \label{tab:NuMuPresel}    
\end{table}
\renewcommand{\arraystretch}{1.0}

\todo[inline]{What does the "selected" variable correspond to.}

\subsection{$\nu_{\mu}$ Selection}
\label{appendix:NuMuSelection}

The muon event selection begins by applying the $\nu_{\mu}$ preselection from Table~\ref{tab:NuMuPresel} and counting the number of tracks which satisfy the criteria listed in Table~\ref{tab:MuonTrackReq}.

\renewcommand{\arraystretch}{1.4}
\begin{table}[H]
    \centering
    \begin{tabular}{c|c}
        \hline
        Cut Description & Query \\
        \hline
        \hline
        \multirow{2}{*}{MIP-like} & \verb|trk_score > 0.8| \\
        & \verb|trk_llr_pid > 0.2| \\
        \hline
        \multirow{6}{*}{Fiducial Volume} & \verb|5 < trk_start_x < 251| \\
        & \verb|5 < trk_end_x < 251| \\
        & \verb|-110 < trk_start_y < 110| \\
        & \verb|-110 < trk_end_y < 110| \\
        & \verb|20 < trk_start_z < 986| \\
        & \verb|20 < trk_end_z < 986| \\
        \hline
        \multirow{4}{*}{Track Quality} & \verb|0.5 > (trk_mcs_muon_mom -| \\ 
        & \verb|trk_range_muon_mom)/trk_range_muon_mom > 0.5| \\
        & \verb|trk_len > 10| \\
        & \verb|trk_distance_v < 4.0| \\
        \hline
        \hline
    \end{tabular}
    \cprotect\caption{Criteria applied to tracks in order to identify them as belonging to muons. All variables shown belong to vector valued branches of the PeLEE ntuple files which can be found with the addition of \verb|_v| to their names in this table (eg. \verb|pfp_generation_v|).}
    \label{tab:MuonTrackReq}
\end{table}
\renewcommand{\arraystretch}{1.0}

\subsection{$\pi^0$ Selection}
\label{appendix:Pi0Selection}

\renewcommand{\arraystretch}{1.4}
\begin{table}[H]
    \centering
    \begin{tabular}{c|c}
        \hline
        Cut Description & Query \\
        \hline
        \hline
        Blindness & \verb|n_showers_contained >= 2| \\
        \hline
        Neutrino slice ID & \verb|nslice == 1| \\
        \hline
        Containment & \verb|contained_fraction > 0.4| \\
        \hline
        Shower Energy & \verb|shr_energy_tot_cali > 0.07| \\
        \hline
        \multirow{2}{*}{Shower-like score cut} & \verb|pi0_shrscore1 < 0.5| \\
         & \verb|pi0_shrscore2 < 0.5| \\
        \hline
        \multirow{2}{*}{Shower-vertex alignment} & \verb|pi0_dot1 > 0.8| \\
        & \verb|pi0_dot2 > 0.8| \\
        \hline
        \multirow{2}{*}{Conversion distance} & \verb|pi0_radlen1 > 3| \\
        & \verb|pi0_radlen2 > 3| \\
        \hline
        \multirow{2}{*}{Energy cut} & \verb|pi0_energy1_Y > 60| \\
        & \verb|pi0_energy2_Y > 40| \\
        \hline
        $dE/dx$ cut & \verb|pi0_dedx1_fit_Y >= 1| \\
        \hline
        {\bf (This isn't in the old technote)} & \verb|pi0_gammadot < 0.94| \\
         \hline
    \end{tabular}
    \caption{The set of cuts applied to generate the $\pi^0$ sample. Blindness is maintained by first removing any events with fewer than two reconstructed showers, creating a selection orothognal to that of the signal region.}
    \label{tab:Pi0Selection}
\end{table}
\renewcommand{\arraystretch}{1.0}

\subsection{1e0p Selection}
\label{appendix:1e0pSelection}

\renewcommand{\arraystretch}{1.4}
\begin{table}[H]
    \centering
    \begin{tabular}{c|c}
        \hline
        \hline
        Cut Name & Query \\
        \hline
        \hline
        \multirow{2}{*}{$\pi^0$ Rejection} & \verb|n_showers_contained == 1| \\
        & \verb|secondshower_Y_nhit < 50| \\
        \hline
        \multirow{10}{*}{Cosmic Rejection} & \verb|CosmicIPAll3D > 10| \\
        & \verb|CosmicDirAll3D > -0.9| \\
        & \verb|CosmicDirAll3D < 0.9| \\
        & \verb|shr_trk_sce_start_y > -100| \\
        & \verb|shr_trk_sce_start_y < 80| \\
        & \verb|shr_trk_sce_end_y > -100| \\
        & \verb|shr_trk_sce_end_y < 100| \\
        & \verb|shr_trk_len < 300| \\
        & \verb|n_tracks_tot == 0 or (n_tracks_tot > 0| \\
        & \verb|and tk1sh1_angle_alltk > -0.9)| \\
        \hline
        \multirow{3}{*}{$\nu_{\mu}$ Rejection} & \verb|shrmoliereavg < 15| \\
        & \verb|subcluster > 4| \\
        & \verb|trkfit < 0.65| \\
        \hline
        0p Selection & \verb|n_tracks_contained == 0| \\
        \hline
        BDT Selection & \verb|bkg_score > 0.72| \\ 
        \hline
        \end{tabular}
    \caption{Cuts used in the 1e0p event selection.}
    \label{tab:1e0pSel}    
\end{table}
\renewcommand{\arraystretch}{1.0}

\subsection{1eNp Selection}
\label{appendix:1eNpSelection}

\renewcommand{\arraystretch}{1.4}
\begin{table}[H]
    \centering
    \begin{tabular}{c|c}
        \hline
        \hline
        Cut Name & Query \\
        \hline
        \hline
        \multirow{1}{*}{$\pi^0$ Rejection} & \verb|n_showers_contained == 1| \\
        \hline
        \multirow{2}{*}{Cosmic Rejection} & \verb|CosmicIPAll3D > 10| \\
        & \verb|shr_trk_len < 300| \\
        \hline
        \multirow{2}{*}{$\nu_{\mu}$ Rejection} & \verb|shrmoliereavg < 9| \\
        & \verb|subcluster > 4| \\
        \hline
        \multirow{9}{*}{Np Selection} & \verb|n_tracks_contained > 0| \\
        & \verb|trkfit < 0.65| \\
        & \verb|trkpid > 0.02| \\
        & \verb|hits_ratio > 0.50| \\
        & \verb|tksh_distance < 6.0| \\
        & \verb|shr_tkfit_nhits_tot > 1| \\
        & \verb|shr_tkfit_dedx_max > 0.5| \\
        & \verb|shr_tkfit_dedx_max < 5.5| \\
        & \verb|tksh_angle > -0.9| \\
        \hline
        \multirow{2}{*}{BDT Selectio}n & \verb|pi0_score > 0.67| \\ 
        & \verb|nonpi0_score > 0.70| \\
        \hline
        \end{tabular}
    \caption{Cuts used in the 1eNp selection.}
    \label{tab:1eNpSel}    
\end{table}
\renewcommand{\arraystretch}{1.0}

